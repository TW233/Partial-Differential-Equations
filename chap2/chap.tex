\ifx\allfiles\undefined
\documentclass[12pt, a4paper,oneside, UTF8]{ctexbook}
\usepackage[dvipsnames]{xcolor}
\usepackage{mathtools}   % 数学公式(mathtools 是 amsmath 的上位替代)
\usepackage{amsthm}    % 定理环境
\usepackage{amssymb}   % 更多公式符号
\usepackage{graphicx}  % 插图
%\usepackage{mathrsfs}  % 数学字体
%\usepackage{newtxtext,newtxmath}
%\usepackage{arev}
\usepackage{kmath,kerkis}
\usepackage{newtxtext}
\usepackage{bbm}
\usepackage{enumitem}  % 列表
\usepackage{geometry}  % 页面调整
%\usepackage{unicode-math}
\usepackage[colorlinks,linkcolor=black]{hyperref}

\usepackage{wrapfig}


\usepackage{ulem}	   % 用于更多的下划线格式,
					   % \uline{}下划线,\uuline{}双下划线,\uwave{}下划波浪线,\sout{}中间删除线,\xout{}斜删除线
					   % \dashuline{}下划虚线,\dotuline{}文字底部加点


\graphicspath{ {flg/},{../flg/}, {config/}, {../config/} }  % 配置图形文件检索目录
\linespread{1.5} % 行高

% 页码设置
\geometry{top=25.4mm,bottom=25.4mm,left=20mm,right=20mm,headheight=2.17cm,headsep=4mm,footskip=12mm}

% 设置列表环境的上下间距
\setenumerate[1]{itemsep=5pt,partopsep=0pt,parsep=\parskip,topsep=5pt}
\setitemize[1]{itemsep=5pt,partopsep=0pt,parsep=\parskip,topsep=5pt}
\setdescription{itemsep=5pt,partopsep=0pt,parsep=\parskip,topsep=5pt}

% 定理环境
% ########## 定理环境 start ####################################
\theoremstyle{definition}
\newtheorem{defn}{\indent 定义}[section]

\newtheorem{lemma}{\indent 引理}[section]    % 引理 定理 推论 准则 共用一个编号计数
\newtheorem{thm}[lemma]{\indent 定理}
\newtheorem{corollary}[lemma]{\indent 推论}
\newtheorem{criterion}[lemma]{\indent 准则}

\newtheorem{proposition}{\indent 命题}[section]
\newtheorem{example}{\indent \color{SeaGreen}{例}}[section] % 绿色文字的 例 ,不需要就去除\color{SeaGreen}{}
\newtheorem*{rmk}{\indent \color{red}{注}}

% 两种方式定义中文的 证明 和 解 的环境:
% 缺点:\qedhere 命令将会失效【技术有限,暂时无法解决】
\renewenvironment{proof}{\par\textbf{证明.}\;}{\qed\par}
\newenvironment{solution}{\par{\textbf{解.}}\;}{\qed\par}

% 缺点:\bf 是过时命令,可以用 textb f等替代,但编译会有关于字体的警告,不过不影响使用【技术有限,暂时无法解决】
%\renewcommand{\proofname}{\indent\bf 证明}
%\newenvironment{solution}{\begin{proof}[\indent\bf 解]}{\end{proof}}
% ######### 定理环境 end  #####################################

% ↓↓↓↓↓↓↓↓↓↓↓↓↓↓↓↓↓ 以下是自定义的命令  ↓↓↓↓↓↓↓↓↓↓↓↓↓↓↓↓

% 用于调整表格的高度  使用 \hline\xrowht{25pt}
\newcommand{\xrowht}[2][0]{\addstackgap[.5\dimexpr#2\relax]{\vphantom{#1}}}

% 表格环境内长内容换行
\newcommand{\tabincell}[2]{\begin{tabular}{@{}#1@{}}#2\end{tabular}}

% 使用\linespread{1.5} 之后 cases 环境的行高也会改变,重新定义一个 ca 环境可以自动控制 cases 环境行高
\newenvironment{ca}[1][1]{\linespread{#1} \selectfont \begin{cases}}{\end{cases}}
% 和上面一样
\newenvironment{vx}[1][1]{\linespread{#1} \selectfont \begin{vmatrix}}{\end{vmatrix}}

\def\d{\textup{d}} % 直立体 d 用于微分符号 dx
\def\R{\mathbb{R}} % 实数域
\def\N{\mathbb{N}} % 自然数域
\def\C{\mathbb{C}} % 复数域
\def\Z{\mathbb{Z}} % 整数环
\def\Q{\mathbb{Q}} % 有理数域
\newcommand{\bs}[1]{\boldsymbol{#1}}    % 加粗,常用于向量
\newcommand{\ora}[1]{\overrightarrow{#1}} % 向量

% 数学 平行 符号
\newcommand{\pll}{\kern 0.56em/\kern -0.8em /\kern 0.56em}

% 用于空行\myspace{1} 表示空一行 填 2 表示空两行  
\newcommand{\myspace}[1]{\par\vspace{#1\baselineskip}}

%s.t. 用\st就能打出s.t.
\DeclareMathOperator{\st}{s.t.}

%罗马数字 \rmnum{}是小写罗马数字, \Rmnum{}是大写罗马数字
\makeatletter
\newcommand{\rmnum}[1]{\romannumeral #1}
\newcommand{\Rmnum}[1]{\expandafter@slowromancap\romannumeral #1@}
\makeatother
\begin{document}
	% \title{{\Huge{\textbf{$Partial \,\, Differential \,\, Equations$}}}\footnote{参考书籍:\\
			\hspace*{4em} \textbf{《Partial Differential Equations》 -- Lawrence C. Evans} \\
			\hspace*{4em} \textbf{《Partial Differential Equations》 -- Fritz John} \\
			\hspace*{4em} \textbf{《数学物理方程讲义 (第二版)》--  姜礼尚、陈亚浙、刘西垣、易法槐} 
			}}
\author{$-TW-$}
\date{\today}
\maketitle                   % 在单独的标题页上生成一个标题

\thispagestyle{empty}        % 前言页面不使用页码
\begin{center}
	\Huge\textbf{序}
\end{center}


\vspace*{3em}
\begin{center}
	\large{\textbf{天道几何,万品流形先自守;}}\\
	
	\large{\textbf{变分无限,孤心测度有同伦。}}
\end{center}

\vspace*{3em}
\begin{flushright}
	\begin{tabular}{c}
		\today \\ \small{\textbf{长夜伴浪破晓梦,梦晓破浪伴夜长}}
	\end{tabular}
\end{flushright}


\newpage                      % 新的一页
\pagestyle{plain}             % 设置页眉和页脚的排版方式(plain:页眉是空的,页脚只包含一个居中的页码)
\setcounter{page}{1}          % 重新定义页码从第一页开始
\pagenumbering{Roman}         % 使用大写的罗马数字作为页码
\tableofcontents              % 生成目录

\newpage                      % 以下是正文
\pagestyle{plain}
\setcounter{page}{1}          % 使用阿拉伯数字作为页码
\pagenumbering{arabic}
\setcounter{chapter}{0}    % 设置 -1 可作为第零章绪论从第零章开始 
	\else
	\fi
	%  ############################ 正文部分
\chapter{Laplace's Equation}
\section{Laplace's Equation}
	这一章我们来学习\textbf{Laplace's Equation}, 这是$PDE$ 中最重要的一类方程. 下面给出\textbf{Laplace's Equation}的定义形式:
	
	\begin{align}
		\begin{cases}
			\Delta u = 0 \hspace*{2em} (\text{Laplace}) \\
			-\Delta u = f \hspace*{2em} (\text{Poisson})
		\end{cases}
	\end{align}
	where $x \in U \underset{open}{\subset} \R^n , \,\, f : U \longrightarrow \R$ is given and $u : \overline{U} \longrightarrow \R$ is the unknown. 
	
	\begin{defn}\label{def 2.1.1}
		A function $u \in C^2$ satisfying $\Delta u = 0$ is called a \underline{\textcolor{blue}{\textbf{harmonic function}}}.
	\end{defn}

	\vspace{4em}
	
	下面给出Laplace's Equation的\textbf{物理含义}. 设$u$ 为$\overline{U} \subset \R^n$ 上的一种物理量(温度、浓度等)在稳态之下的分布情况. $F : U \longrightarrow \R^n$ 表示该物理量的流量密度. 以$u$ 为浓度为例, 若某一点附近的浓度比该点低, 则会有浓度跑向于该点附近的趋势, 此时$F$ 在该点处的值不为零.
	
	\vspace{1em}
	
	而\textbf{Laplace's Equation}则描述了空间$U$ 中各处物理量$u$ 均维持不变一种稳态, 即对于$\forall V \subset \subset U$, $F$ 在$\partial V$ 上的通量为0, 即
	\[ \int_{\partial V} \vec{F} \cdot \vec{\gamma} = 0 \]
	根据\textbf{Gauss-Green公式 (Thm \ref{thm 1.6.1})}, 得到
	\[ \int_{V} div \, \vec{F} = \int_{\partial V} \vec{F} \cdot \vec{\gamma} = 0 \]
	根据$V$ 的任意性可知, $div \, \vec{F} = 0$. 而由物理含义, 我们不妨假设$F \propto -Du$, 即物理量总是流向附近较低点, $F = -a Du \,\, (a > 0)$. 则
	\[ div \, \vec{F} = 0 \,\, \Rightarrow \,\, div(Du) = \Delta u = 0 \]
	
\newpage
\subsection{Laplace's Equation的基本解}
	类比$ODE$ 中的\textbf{基础解系}(可线性表出任一解), 对于$PDE$, 我们通常会在原方程基础上加以限制, 得到性质较好的解, 再用该解来表出其他解(不一定线性). 这样的解我们称为\textbf{基本解}. 
	
	\vspace{1em}
	
	特别地, 对于\textbf{Laplace's Equation}, 我们要求基本解函数满足\textbf{旋转对称性 (rotation invariance)}, 即$u(x) = \widetilde{u} (\Vert x \Vert)$, 即$u$ 的取值只与自变量$x$ 的模长有关系, 在以原点为球心的任一球面上的函数值相等. 下面来对基本解进行求解. 
	
	\vspace{4em}
	
	Let $r \coloneqq \left| x \right| = \sqrt{x_{1}^2 + \cdots + x_{n}^2}$, then $\dfrac{\partial r}{\partial x_i} = \dfrac{x_i}{\sqrt{x_{1}^2 + \cdots + x_{n}^2}} = \dfrac{x_i}{r}$. \\
	Suppose $u(x) = v(r)$, thus
	\begin{align}
		\Delta u 
		= \sum_{i = 1}^{n} u_{x_i x_i} 
		= \sum_{i = 1}^{n} \left( \frac{\partial}{\partial x_i} \right)^2 v(r) 
		= \sum_{i = 1}^{n} \frac{\partial}{\partial x_i} \left( v^{'}(r) \frac{\partial r}{\partial x_i} \right) 
		&= \sum_{i = 1}^{n} \frac{\partial}{\partial x_i} \left( v^{'}(r) \frac{x_i}{r} \right) \\
		&= \sum_{i = 1}^{n} \left( v^{''}(r) \frac{x_{i}^2}{r^2} + v^{'}(r) \cfrac{r - x_i \frac{x_i}{r}}{r^2} \right) \\
		&= v^{''}(r) + \frac{n - 1}{r} v^{'}(r) = 0
	\end{align}
	If $v^{'} \neq 0$, then 
	\[ \left( \log \left| v^{'} \right| \right)^{'} = \frac{v^{''}}{v^{'}} = \frac{1 - n}{r} \]
	最终解得
	\begin{align}
		v = 
		\begin{cases}
			b \log r + c , \,\, n = 2 \\
			\dfrac{b}{r^{n - 2}} + c , \,\, n \geq 3
		\end{cases} , \,\, b , c \in \R \,\, \text{const}
	\end{align}
	最后我们固定$b , c$ 的取值, 得到\textbf{Laplace's Equation的基本解}. 
	
	\vspace{2em}
	\begin{defn}\label{def 2.1.2}
		The function
		\begin{align}
			\Phi(x) \coloneqq 
			\begin{cases}
				-\cfrac{1}{2 \pi} \log \left| x \right| \hspace*{2em} (n = 2) \\
				\cfrac{1}{n (n - 2) \alpha(n)} \cdot \cfrac{1}{\left| x \right|^{n - 2}} \hspace*{2em} (n \geq 3)
			\end{cases} , \,\, x \in \R^n \, \backslash \, \{ 0 \}
		\end{align}
		is the \underline{\textcolor{blue}{\textbf{fundamental solution of Laplace's Equation}}}.
	\end{defn}

\newpage

\subsection{求解Possion方程}
	下面我们就用\textbf{Laplace方程的基本解}来给出\textbf{Possion方程}的一个解, 并且我们将在后面证明, 在忽略常数的情况下这就是\textbf{Possion方程}的所有解. 

	\vspace{1em}

	\begin{thm}\label{thm 2.1.1}
		\textbf{[Solving Possion's Equation using $\Phi$]}. \\
		Assume $f \in C_{c}^2 (\R^n)$ and $u(x) \coloneqq \int_{\R^n} \Phi(x - y) f(y) \, dy$. Then
		\begin{enumerate}
			\item[(\rmnum{1})] $u \in C^2 (\R^n)$. 
		
			\item[(\rmnum{2})] $- \Delta u = f$ in $\R^n$. 
		\end{enumerate}
		
		\vspace{4em}
		
		\begin{proof}
			\begin{enumerate}
				\item[(\rmnum{1})] By \textbf{卷积的对称性}, 
				\[ u(x) = \int_{\R^n} \Phi(x - y) f(y) \, dy = \int_{\R^n} \Phi(y) f(x - y) \, dy \]
				Since $\Phi \in C(\R^n)$ and $f \in C_{c}^2 (\R^n)$, then by \textbf{卷积的可微性 (Prop \ref{prop A.5.1})}, $u \in C^2 (\R^n)$ and
				\[ 
				\begin{cases}
					u_{x_i} = \Phi * \dfrac{\partial}{\partial x_i} f \\
					u_{x_i x_i} = \Phi * \dfrac{\partial^2}{\partial x_{i}^2} f
				\end{cases} , \,\, \forall 1 \leq i \leq n
				\]
				
				\vspace{2em}
				
				\item[(\rmnum{2})] 首先先来定性分析. \\
				根据\textbf{$\Phi(x)$ 的定义 (\ref{def 2.1.2})}, $\Phi$ 在原点附近趋于$+ \infty$, 而$f \in C_{c}^2 (\R^n)$ 只在有界区域上有有限值, 因此根据(\rmnum{1}), 
				\[ \Delta u(x) = \int_{\R^n} \Phi(y) \Delta_x f(x - y) \, dy \]
				我们应当分别对于$0$ 附近的区域$B(0 , \epsilon)$ 和以外的区域$B(0 , \epsilon)^c$ 进行讨论. i.e.
				\begin{align}
					\Delta u(x) 
					&= \int_{B(0 , \epsilon)} \Phi(y) \Delta_x f(x - y) \, dy + \int_{B(0 , \epsilon)^c} \Phi(y) \Delta_x f(x - y) \, dy \\
					&\coloneqq I_\epsilon + J_\epsilon
				\end{align}
				下面分别对$I_\epsilon$ 和$J_\epsilon$ 进行讨论. 
				\begin{itemize}
					\item Since $f \in C_{c}^2 (\R^n)$, then $\dfrac{\partial^2}{\partial x_{i}^2} f \in D^2 f \subset C_{c}(\R^n) , \,\, \forall 1 \leq i \leq n$. \\
					Since $D^2 f$ 中元素个数有限, then $\exists M > 0$, $\st$ 
					\[ \Vert D^2 f \Vert_{L^\infty (\R^n)} \leq M , \,\, \forall x \in \R^n \]
					Thus $\left| \Delta_x f(x - y) \right| \leq C \cdot M , \,\, \forall x , y \in \R^n$ for some $C > 0$. Then
					\begin{align}
						\left| I_\epsilon \right| 
						\leq C \cdot \Vert D^2 f \Vert_{L^\infty (\R^n)} \int_{B(0 , \epsilon)} \left| \Phi(y) \right| \, dy
					\end{align}
					由于$\Phi(x)$ 具有\textbf{辐向对称性 (Def \ref{def 2.1.2})}, 故$\left| \Phi(x) \right|$ 同样具有辐向对称性, 即$\widetilde{\Phi}(\Vert y \Vert) = \left| \Phi(y) \right|$, $\left| \Phi \right|$ 的取值只与自变量的模长有关, 于是
					\[ \int_{B(0 , \epsilon} \left| \Phi(y) \right| \, dy = \int_{B(0 , \epsilon)} \widetilde{\Phi} (r) \, dy \]
					根据\textbf{球坐标变换 (附录 \ref{appendix A} - Def \ref{def A.6.1} $\&$ Example \ref{ex A.6.1})}, 
					\[
						\int_{B(0 , \epsilon)} \left| \Phi(y) \right| \, dy 
						= \int_{B(0 , \epsilon)} \widetilde{\Phi} (r) \, dy 
						= n \alpha(n) \cdot \int_{0}^\epsilon r^{n - 1} \widetilde{\Phi} (r) \, dr
					\]
					By the \textbf{definition of $\Phi$ (Def \ref{def 2.1.2})}, 
					\begin{align}
						\Phi(x) \coloneqq 
						\begin{cases}
							-\cfrac{1}{2 \pi} \log \left| x \right| \hspace*{2em} (n = 2) \\
							\cfrac{1}{n (n - 2) \alpha(n)} \cdot \cfrac{1}{\left| x \right|^{n - 2}} \hspace*{2em} (n \geq 3)
						\end{cases} , \,\, x \in \R^n \, \backslash \, \{ 0 \}
					\end{align}
					Since
					\begin{align}
						r^{n - 1} \widetilde{\Phi}(r) = 
						\begin{cases}
							C_1 r^{n - 1} \left| \log r \right| \\
							C_2 \dfrac{r^{n - 1}}{r^{n - 2}} = C_2 r
						\end{cases} \to 0 \,\, \text{as} \,\, r \to 0^+
					\end{align}
					Then 
					\[ \int_{B(0 , \epsilon)} \left| \Phi(y) \right| \, dy 
					= n \alpha(n) \cdot \int_{0}^\epsilon r^{n - 1} \widetilde{\Phi} (r) \, dr \to 0 \,\, \text{as} \,\, \epsilon \to 0^+ \]
					Therefore, 
					\[ \left| I_\epsilon \right| 
					\leq C \cdot \Vert D^2 f \Vert_{L^\infty (\R^n)} \int_{B(0 , \epsilon)} \left| \Phi(y) \right| \, dy \to 0 \,\, \text{as} \,\, \epsilon \to 0^+ \]
					故当$\epsilon \to 0^+$ 时, 
					\[ \Delta u(x) = J_\epsilon \]
					
					\newpage
					
					\item Since
					\begin{align}
						\Delta_x f(x - y) 
						= \sum_{i = 1}^{n} \left( \frac{\partial}{\partial x_i} \right)^2 f(x - y) 
						&= \sum_{i = 1}^n \left( \frac{\partial}{\partial (x_i - y_i)} \cdot \frac{\partial (x_i - y_i)}{\partial x_i} \right)^2 f(x - y) \\
						&= \sum_{i = 1}^n \left( - \frac{\partial}{\partial (y_i - x_i)} \right)^2 f(x - y) \\
						&= \sum_{i = 1}^n \left( \frac{\partial}{\partial y_i} \right)^2 f(x - y) \\
						&= \Delta_y f(x - y)
					\end{align}
					Then by \textbf{Green's Formula (Cor \ref{cor 1.6.3})}, 
					\begin{align}
						J_\epsilon 
						&= \int_{B(0 , \epsilon)^c} \Phi(y) \Delta_y f(x - y) \, dy \\
						&= -\int_{B(0 , \epsilon)^c} \nabla \Phi(y) \cdot \nabla_y f(x - y) \, dy + \int_{\partial \left( B(0 , \epsilon) \right)^c} \Phi(y) \frac{\partial f}{\partial \vec{\gamma}}(x - y) \, dS(y) \\
						&\coloneqq K_\epsilon + L_\epsilon
					\end{align}
					下面分别对$K_\epsilon$ 和$L_\epsilon$ 进行讨论. 首先是$L_\epsilon$:\\
					Since 
					\[ \Phi(y) = 
						\begin{cases}
							-\cfrac{1}{2 \pi} \log \epsilon \hspace*{2em} (n = 2) \\
							\cfrac{1}{n (n - 2) \alpha(n)} \cdot \cfrac{1}{\epsilon^{n - 2}} \hspace*{2em} (n \geq 3)
						\end{cases} , \,\, \forall y \in \partial \left( B(0 , \epsilon) \right)^c
					 \]
					 Then
					\begin{align}
						\left| L_\epsilon \right| 
						&\leq C \cdot \Vert D f \Vert_{\infty} \int_{\partial \left( B(0 , \epsilon) \right)^c} \left| \Phi(y) \right| \, d S(y) \\
						&= 
						\begin{cases}
							C_1 \cdot \epsilon \left| \log \epsilon \right| , \,\, n = 2 \\
							C_2 \cdot \cfrac{\epsilon^{n - 1}}{\epsilon^{n - 2}} = C_2 \epsilon , \,\, n \geq 3
						\end{cases} \to 0 \,\, \text{as} \,\, \epsilon \to 0^+
					\end{align}
					So when $\epsilon \to 0^+$, $J_\epsilon = K_\epsilon$. 
					
					\vspace{1em}
					
					By \textbf{Green's Formula (Cor \ref{cor 1.6.3})}, 
					\begin{align}
						K_\epsilon 
						&= -\int_{B(0 , \epsilon)^c} \nabla \Phi(y) \cdot \nabla_y f(x - y) \, dy \\
						&= \int_{B(0 , \epsilon)^c} \Delta \Phi(y) f(x - y) \, dy - \int_{\partial \left( B(0 , \epsilon) \right)^c} \frac{\partial \Phi}{\partial \vec{\gamma}}(y) f(x - y) \, dS(y)
					\end{align}
					Since $\Delta \Phi \equiv 0$, then 
					\[ K_\epsilon = - \int_{\partial \left( B(0 , \epsilon) \right)^c} \frac{\partial \Phi}{\partial \vec{\gamma}}(y) f(x - y) \, dS(y) 
					= \int_{\partial B(0 , \epsilon)} \frac{\partial \Phi}{\partial \vec{\gamma}} (y) f(x - y) \, dS(y) \]
					Since $\vec{\gamma}$ 为$\partial B(0 , \epsilon)$ 的\textbf{单位外法向 (附录 \ref{appendix A} - Def \ref{def A.1.1})}, $\Phi(y)$ 径向对称
					\[ 
					 \Phi(y) = 
					 \begin{cases}
					 	-\cfrac{1}{2\pi} \log \left| y \right| , \,\, n = 2 \\
					 	\cfrac{1}{n (n - 2) \alpha(n)} \cdot \cfrac{1}{\left| y \right|^{n - 2}} , \,\, n \geq 3
					 \end{cases} 
				 	\]
				 	Then $\dfrac{\partial \Phi(y)}{\partial \vec{\gamma}}$ 即为$\Phi$ 在$y$ 点沿法向$\vec{\gamma}$ 的方向导数, i.e. 
				 	\[ 
				 	 \frac{\partial \Phi (y)}{\partial \vec{\gamma}} = 
				 	 \begin{cases}
				 	 	-\cfrac{1}{2\pi} \cdot \cfrac{1}{\left| y \right|} , \,\, n = 2 \\
				 	 	- \cfrac{1}{n \alpha(n)} \cdot \cfrac{1}{\left| y \right|^{n - 1}} , \,\, n \geq 3
				 	 \end{cases} \,\, \Rightarrow \,\, 
			 	 	\frac{\partial \Phi (y)}{\partial \vec{\gamma}} \Big|_{\left| y \right| = \epsilon} = 
			 	 	\begin{cases}
			 	 		-\cfrac{1}{2\pi} \cdot \cfrac{1}{\epsilon} , \,\, n = 2 \\
			 	 		- \cfrac{1}{n \alpha(n)} \cdot \cfrac{1}{\epsilon^{n - 1}} , \,\, n \geq 3
			 	 	\end{cases} 
		 	 		= - \frac{1}{\mu \left( \partial B(0 , \epsilon) \right)}
			 	 	\]
			 	 	于是
			 	 	\begin{align}
			 	 		K_\epsilon 
			 	 		&= -\frac{1}{\mu \left( \partial B(0 , \epsilon) \right)} \int_{\partial B(0 , \epsilon)} f(x - y) \, dS(y) \\
			 	 		&= - \int_{\partial B(0 , \epsilon)}\hspace{-2.70em}-\ \hspace*{2.05em} f(x) \, d S(y) \to - f(x) \,\, \text{as} \,\, \epsilon \to 0^+
			 	 	\end{align}
				\end{itemize}
				
				\vspace{1em}
				
				Shortly speaking, 
				\[
					\Delta u(x) = I_\epsilon + J_\epsilon 
					= I_\epsilon + L_\epsilon + K_\epsilon
				\]
				Letting $\epsilon \to 0^+$, then $I_\epsilon , L_\epsilon \to 0$ and $K_\epsilon \to -f(x)$, i.e.
				\[ \Delta u(x) = -f(x) \]
				Therefore, $-\Delta u(x) = f(x)$ in $\R^n$.
			\end{enumerate}
		\end{proof}
	\end{thm}



	%  ############################
	\ifx\allfiles\undefined
\end{document}
\fi