\ifx\allfiles\undefined
\input{../config/config}
\begin{document}
	% \input{../config/cover} 
	\else
	\fi
	%  ############################ 正文部分
\chapter{Laplace's Equation}
\section{Laplace's Equation}
	这一章我们来学习\textbf{Laplace's Equation}, 这是$PDE$ 中最重要的一类方程. 下面给出\textbf{Laplace's Equation}的定义形式:
	
	\begin{align}
		\begin{cases}
			\Delta u = 0 \hspace*{2em} (\text{Laplace}) \\
			-\Delta u = f \hspace*{2em} (\text{Poisson})
		\end{cases}
	\end{align}
	where $x \in U \underset{open}{\subset} \R^n , \,\, f : U \longrightarrow \R$ is given and $u : \overline{U} \longrightarrow \R$ is the unknown. 
	
	\begin{defn}\label{def 2.1.1}
		A function $u \in C^2$ satisfying $\Delta u = 0$ is called a \underline{\textcolor{blue}{\textbf{harmonic function}}}.
	\end{defn}

	\vspace{4em}
	
	下面给出Laplace's Equation的\textbf{物理含义}. 设$u$ 为$\overline{U} \subset \R^n$ 上的一种物理量(温度、浓度等)在稳态之下的分布情况. $F : U \longrightarrow \R^n$ 表示该物理量的流量密度. 以$u$ 为浓度为例, 若某一点附近的浓度比该点低, 则会有浓度跑向于该点附近的趋势, 此时$F$ 在该点处的值不为零.
	
	\vspace{1em}
	
	而\textbf{Laplace's Equation}则描述了空间$U$ 中各处物理量$u$ 均维持不变一种稳态, 即对于$\forall V \subset \subset U$, $F$ 在$\partial V$ 上的通量为0, 即
	\[ \int_{\partial V} \vec{F} \cdot \vec{\gamma} = 0 \]
	根据\textbf{Gauss-Green公式 (Thm \ref{thm 1.6.1})}, 得到
	\[ \int_{V} div \, \vec{F} = \int_{\partial V} \vec{F} \cdot \vec{\gamma} = 0 \]
	根据$V$ 的任意性可知, $div \, \vec{F} = 0$. 而由物理含义, 我们不妨假设$F \propto -Du$, 即物理量总是流向附近较低点, $F = -a Du \,\, (a > 0)$. 则
	\[ div \, \vec{F} = 0 \,\, \Rightarrow \,\, div(Du) = \Delta u = 0 \]
	
\newpage
\subsection{Laplace's Equation的基本解}
	类比$ODE$ 中的\textbf{基础解系}(可线性表出任一解), 对于$PDE$, 我们通常会在原方程基础上加以限制, 得到性质较好的解, 再用该解来表出其他解(不一定线性). 这样的解我们称为\textbf{基本解}. 
	
	\vspace{1em}
	
	特别地, 对于\textbf{Laplace's Equation}, 我们要求基本解函数满足\textbf{旋转对称性 (rotation invariance)}, 即$u(x) = \widetilde{u} (\Vert x \Vert)$, 即$u$ 的取值只与自变量$x$ 的模长有关系, 在以原点为球心的任一球面上的函数值相等. 下面来对基本解进行求解. 
	
	\vspace{4em}
	
	Let $r \coloneqq \left| x \right| = \sqrt{x_{1}^2 + \cdots + x_{n}^2}$, then $\dfrac{\partial r}{\partial x_i} = \dfrac{x_i}{\sqrt{x_{1}^2 + \cdots + x_{n}^2}} = \dfrac{x_i}{r}$. \\
	Suppose $u(x) = v(r)$, thus
	\begin{align}
		\Delta u 
		= \sum_{i = 1}^{n} u_{x_i x_i} 
		= \sum_{i = 1}^{n} \left( \frac{\partial}{\partial x_i} \right)^2 v(r) 
		= \sum_{i = 1}^{n} \frac{\partial}{\partial x_i} \left( v^{'}(r) \frac{\partial r}{\partial x_i} \right) 
		&= \sum_{i = 1}^{n} \frac{\partial}{\partial x_i} \left( v^{'}(r) \frac{x_i}{r} \right) \\
		&= \sum_{i = 1}^{n} \left( v^{''}(r) \frac{x_{i}^2}{r^2} + v^{'}(r) \cfrac{r - x_i \frac{x_i}{r}}{r^2} \right) \\
		&= v^{''}(r) + \frac{n - 1}{r} v^{'}(r) = 0
	\end{align}
	If $v^{'} \neq 0$, then 
	\[ \left( \log \left| v^{'} \right| \right)^{'} = \frac{v^{''}}{v^{'}} = \frac{1 - n}{r} \]
	最终解得
	\begin{align}
		v = 
		\begin{cases}
			b \log r + c , \,\, n = 2 \\
			\dfrac{b}{r^{n - 2}} + c , \,\, n \geq 3
		\end{cases} , \,\, b , c \in \R \,\, \text{const}
	\end{align}
	最后我们固定$b , c$ 的取值, 得到\textbf{Laplace's Equation的基本解}. 
	
	\vspace{2em}
	\begin{defn}\label{def 2.1.2}
		The function
		\begin{align}
			\Phi(x) \coloneqq 
			\begin{cases}
				-\cfrac{1}{2 \pi} \log \left| x \right| \hspace*{2em} (n = 2) \\
				\cfrac{1}{n (n - 2) \alpha(n)} \cdot \cfrac{1}{\left| x \right|^{n - 2}} \hspace*{2em} (n \geq 3)
			\end{cases} , \,\, x \in \R^n \, \backslash \, \{ 0 \}
		\end{align}
		is the \underline{\textcolor{blue}{\textbf{fundamental solution of Laplace's Equation}}}.
	\end{defn}



	%  ############################
	\ifx\allfiles\undefined
\end{document}
\fi