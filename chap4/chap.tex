\ifx\allfiles\undefined
\input{../config/config}
\begin{document}
	% \input{../config/cover} 
	\else
	\fi
	%  ############################ 正文部分
\chapter{Hyperbolic Equations in Higher Dimensions}

	\begin{center}
		这一章我们将给出\textbf{高维波动方程的解}, 并利用\textbf{高维能量不等式}证明其解的唯一性. 
	\end{center}
	
\section{Euler-Darboux-Poisson Equation}
	
	\vspace*{2em}
	
	\begin{thm}\label{thm 4.1.1}
		\textbf{[Euler-Darboux-Poisson Equation]}. \\
		Suppose $u \in C^2\Big( \R^n \times (0 , \infty) \Big) \cap C \Big( \R^n \times [0 , \infty) \Big)$ satisfies 
		\begin{align*}
			u_{tt} - c^2 \Delta u = 0 \,\, \text{on} \,\, \R^n \times (0 , \infty)
		\end{align*}
		Then 
		\begin{align}
			c^2 \left[ \frac{\partial^2}{\partial r^2} M_{u}(x ; r , t) + \frac{n - 1}{r} \frac{\partial}{\partial r} M_u(x ; r , t) \right] 
			= \frac{\partial^2}{\partial t^2} M_u(x ; r , t) \label{4.1}
		\end{align}
		
		\vspace*{4em}
		
		\begin{rmk}
			\begin{itemize}
				\item 此处给出了\textbf{球面平均}的记号:$\forall h \in C(\R^n)$, 
				\begin{align*}
					M_h(x , r) \coloneqq \fint_{\partial B_{r}(x)} h(y) \, dS_y
				\end{align*}
				而\textbf{式(\ref{4.1})}中的“$M_u(x ; r , t)$”则是将$x$ 固定, $r , t$ 作为变量, i.e. 
				\begin{align*}
					M_u(x ; r , t) 
					= \fint_{\partial B_r(x)} u(y , t) \, dS_y 
					= \fint_{\partial B_r} u(x + y , t) \, dS_y
				\end{align*}
				
				\vspace*{2em}
				
				\item 利用\textbf{E-D-P公式}可求得任意\textbf{奇数维}波动方程的解. 
			\end{itemize}
		\end{rmk}
		
		\newpage
		
		\begin{proof}
			$\forall h \in C^2(\R^n)$, 
			\begin{align*}
				M_h(x , r) 
				&= \fint_{\partial B_r(x)} h(y) \, dS_y 
				= \frac{1}{w_n r^{n - 1}} \int_{\partial B_r} h(x + y) \, dS_y 
				= \frac{1}{w_n} \int_{\partial B_1} h(x + ry) \, dS_y \\
				&\left( = \frac{1}{w_n} \int_{\partial B_1} h(x - ry) \, dS_y , \,\, \text{事实上此时即可将$r$ 的定义域由$(0 , \infty)$ 扩充至$\R$ } \right)
			\end{align*}
			将$M_h(x , r)$ 对$r$ 求偏导, 得到
			\begin{align*}
				\frac{\partial}{\partial r} M_h(x , r) 
				&= \frac{\partial}{\partial r} \frac{1}{w_n} \int_{\partial B_1} h(x + ry) \, dS_y \\
				&= \frac{1}{w_n} \int_{\partial B_1} \left( \sum_{i = 1}^n h_i(x + ry) \cdot y_i \right) \, dS_y 
				= \frac{1}{w_n} \int_{\partial B_1} \frac{1}{r} \cdot \frac{\partial h(x + ry)}{\partial n_y} \, dS_y
			\end{align*}
			根据\textbf{Gauss-Green公式 (Cor \ref{cor B.4.3} (\rmnum{1}))}, 
			\begin{align*}
				\frac{\partial}{\partial r} M_h(x , r) 
				&= \frac{1}{w_n} \int_{B_1} \frac{1}{r} \cdot \Delta_y h(x + ry) \, dy \\
				&= \frac{1}{w_n} \int_{B_1} r \cdot \Delta_{(ry)} h(x + ry) \, dy \\
				&= \frac{1}{w_n} \int_{B_1} r \cdot \Delta_{x} h(x + ry) \, dy \\
				&= \Delta_{x} \frac{r}{w_n} \int_{B_1} h(x + ry) \, dy \\
				&= \Delta_x \frac{r}{w_n} \int_{B_r} h(x + y) \cdot \frac{1}{r^n} \, dy 
				= \Delta_x r^{1 - n} \cdot \frac{1}{w_n} \int_{B_r} h(x + y) \, dy
			\end{align*}
			i.e. 
			\begin{align*}
				r^{n - 1} \frac{\partial}{\partial r} M_h(x , r) 
				= \frac{1}{w_n} \Delta_x \int_{B_r} h(x + y) \, dy
			\end{align*}
			两侧对$r$ 求偏导, 
			\begin{align*}
				\frac{\partial}{\partial r} \left( r^{n - 1} \frac{\partial}{\partial r} M_h(x , r) \right) 
				= \frac{1}{w_n} \Delta_x \int_{\partial B_r} h(x + y) \, dS_y
			\end{align*}
			两侧除以$r^{n - 1}$, 右侧可得到球面平均的形式, 即
			\begin{align*}
				\frac{1}{r^{n - 1}} \frac{\partial}{\partial r} \left( r^{n - 1} \frac{\partial}{\partial r} M_h(x , r) \right) 
				= \Delta_x M_h(x , r)
			\end{align*}
			i.e. 
			\begin{align*}
				\frac{\partial^2}{\partial r^2} M_h(x , r) + \frac{n - 1}{r} \frac{\partial}{\partial r} M_h(x , r) 
				= \Delta_x M_h(x , r)
			\end{align*}
			
			\newpage
			
			对于$u(x , t) \in C^2\Big( \R^n \times (0 , \infty) \Big) \cap C \Big( \R^n \times [0 , \infty) \Big)$ with 
			\begin{align*}
				u_{tt} - c^2 \Delta u = 0 \,\, \text{on} \,\, \R^n \times (0 , \infty)
			\end{align*}
			Fix $t \in (0 , \infty)$. 将$u(x ; t)$ 替换$h(x)$, 计算前文得到的等式右侧式子
			\begin{align*}
				\Delta_x M_u(x ; r , t) 
				&= \Delta_x \fint_{\partial B_r(x)} u(y , t) \, dS_y \\
				&= \Delta_x \fint_{\partial B_r} u(x + y , t) \, dS_y \\
				&= \fint_{\partial B_r} \Delta_x u(x + y , t) \, dS_y \\
				&= \frac{1}{c^2} \fint_{\partial B_r} u_{tt} (x + y , t) \, dS_y 
				= \frac{1}{c^2} \frac{\partial^2}{\partial t^2} M_{u}(x ; r , t)
			\end{align*}
			代入$\dfrac{\partial^2}{\partial r^2} M_h(x , r) + \dfrac{n - 1}{r} \dfrac{\partial}{\partial r} M_h(x , r) 
			= \Delta_x M_h(x , r)$, 得到
			\begin{align*}
				c^2 \left[ \frac{\partial^2}{\partial r^2} M_{u}(x ; r , t) + \frac{n - 1}{r} \frac{\partial}{\partial r} M_u(x ; r , t) \right] 
				= \frac{\partial^2}{\partial t^2} M_u(x ; r , t)
			\end{align*}
		\end{proof}
	\end{thm}
	
\newpage
	
\section{Kirchhoff Formula (n = 3时波动方程初值问题求解)}
	
	\vspace*{2em}
	
	这一节我们通过\textbf{E-D-P Equation (Thm \ref{thm 4.1.1})}, 来求解3维波动方程的初值问题, 并且结合后续的\textbf{高维能量不等式 (Thm \ref{thm})}, 可得到解的唯一性. 
	
	\vspace*{10em}
	
	\begin{thm}\label{thm 4.2.1}
		\textbf{[Kirchhoff Formula (n = 3时波动方程初值问题求解)]}. \\
		Suppose $f \in C^3(\R^3) , \,\, g \in C^2(\R^3)$. 对于初值问题 
		\begin{align*}
			\begin{dcases*}
				u_{tt} - c^2 \Delta u = 0 \,\, \text{on} \,\, \R^3 \times (0 , \infty) \\
				u \Big|_{t = 0} = f \\
				u_t \Big|_{t = 0} = g
			\end{dcases*}
		\end{align*}
		Then  
		\begin{align}
			u(x , t) 
			= \frac{\partial}{\partial t} \left( \frac{1}{4 \pi c^2 t} \int_{\partial B_{ct}(x)} f(y) \, dS_y \right) + \frac{1}{4 \pi c^2 t} \int_{\partial B_{ct}(x)} g(y) \, dS_y \label{4.2}
		\end{align}
		is a unique solultion in $C^2 \Big( \R^n \times (0 , \infty) \Big) \cap C\Big( \R^n \times [0 , \infty) \Big)$. 
		
		\newpage
		
		\begin{proof}
			\begin{center}
				\textbf{[思路:Step 1:先证明解若存在则必为式(\ref{4.2})的形式. \\
					\hspace{4em}Step 2:证明式(\ref{4.2})满足波动方程及初值条件. ]}
			\end{center}
			
			\vspace*{1em}
			
			\begin{enumerate}
				\item[\underline{\textbf{Step 1}}]:\underline{\textbf{证明解若存在则必为式(\ref{4.2})的形式}}:\\
				For $\forall h(r , t) \in C^2(\R \times (0 , \infty))$, 通过计算可得
				\begin{align*}
					\frac{\partial^2}{\partial r^2} h + \frac{2}{r} \frac{\partial}{\partial r} h 
					= \left[ \frac{\partial^2}{\partial r^2} \Big( rh \Big) \right] \cdot \frac{1}{r}
				\end{align*}
				根据\textbf{E-D-P等式 (Thm \ref{thm 4.1.1})}, 将E-D-P等式两侧同时乘$r$, 可得
				\begin{align*}
					\frac{\partial^2}{\partial t^2} \Big( r h \Big) 
					= c^2 r \left( \frac{\partial^2}{\partial r^2} h + \frac{2}{r} \frac{\partial}{\partial r} h \right) 
					= c^2 \frac{\partial^2}{\partial r^2} \Big( r h \Big) 
				\end{align*}
				将$h(r , t)$ 替换为$M_u(x ; r , t)$, i.e.
				\begin{align*}
					\frac{\partial^2}{\partial t^2} \Big( r M_u(x ; r , t) \Big) 
					= c^2 \frac{\partial^2}{\partial r^2} \Big( r M_u(x ; r , t) \Big)
				\end{align*}
				即$r M_u(x ; r , t)$ 关于变量$r , t$ 满足一维波动方程. 若已知其满足初始条件:
				\begin{align*}
					\left. \Big[ r M_u(x ; r , t) \Big] \right|_{t = 0} 
					= F(x , r) , \,\, 
					\left. \Big[ rM_{u}(x ; r , t) \Big]_t \right|_{t = 0} 
					= G(x , r)
				\end{align*}
				则根据\textbf{D'Alembert公式 (Thm \ref{thm 2.3.1})}, 即可求解得到:
				\begin{align*}
					rM_{u}(x ; r , t) 
					&= \frac{1}{2} \Big[ F(x , r + ct) + F(x , r - ct) \Big] + \frac{1}{2c} \int_{r - ct}^{r + ct} G(x , \xi) \, d\xi \\
					M_{u}(x ; r , t) 
					&= \frac{1}{2r} \Big[ F(x , r + ct) + F(x , r - ct) \Big] + \frac{1}{2cr} \int_{r - ct}^{r + ct} G(x , \xi) \, d\xi
				\end{align*}
				下面分别进行两步操作:\underline{\textbf{(1).假定F, G已知, 求出u;(2).求出F, G.}}
				
				\vspace*{1em}
				
				\begin{enumerate}
					\item[\textbf{(1)}]. \textbf{\underline{假定$F , G$ 已知 (根据后续计算, 可知$F , G$ 关于$r$ 为奇函数)}}:
					\begin{align*}
						M_{u}(x ; r , t) 
						= \fint_{\partial B_{r}(x)} u(y , t) \, dS_y \to u(x , t) \,\, \text{as} \,\, r \to 0^+
					\end{align*}
					
					\begin{itemize}
						\item 由于$F$ 关于$r$ 为奇函数 (根据步(2)计算可得), 因此
						\begin{align*}
							F(x , r - ct) = -F(x , ct - r)
						\end{align*}
						于是
						\begin{align*}
							\frac{F(x , r + ct) - F(x , ct - r)}{2r} 
							= D_2 F(x , ct) 
							= \left. \frac{\partial}{\partial r} F(x , r) \right|_{r = ct}
						\end{align*}
						
						\newpage
						
						\item 对于$\underset{r \to 0^+}{\lim} \dfrac{1}{2cr} \int_{r - ct}^{r + ct} G(x , \xi) \, d\xi$, For fixed $c , t$, when $| r |$ sufficiently small, $r - ct < 0$, 此时
						\begin{align*}
							\int_{r - ct}^{r + ct} \sim 
							= \int_{ct - r}^{ct + r} \sim + \int_{r - ct}^{ct - r} \sim
						\end{align*}
						由于$G$ 为奇函数, 因此$\int_{r - ct}^{r + ct} G(x , \xi) \, d\xi = 0$, 
						\begin{align*}
							\int_{r - ct}^{r + ct} G(x , \xi) \, d\xi 
							= \int_{ct - r}^{ct + r} G(x , \xi) \, d\xi
						\end{align*}
						从而
						\begin{align*}
							\lim_{r \to 0^+} \frac{1}{2cr} \int_{r - ct}^{r + ct} G(x , \xi) \, d\xi 
							= \lim_{r \to 0^+} \frac{1}{2cr} \int_{ct - r}^{ct + r} G(x , \xi) \, d\xi 
							= \frac{1}{c} G(x , ct)
						\end{align*}
					\end{itemize}
					
					\vspace*{2em}
					
					综上, 根据
					\begin{align*}
						M_{u}(x ; r , t) 
						= \frac{1}{2r} \Big[ F(x , r + ct) + F(x , r - ct) \Big] + \frac{1}{2cr} \int_{r - ct}^{r + ct} G(x , \xi) \, d\xi
					\end{align*}
					Letting $r \to 0^+$, then 
					\begin{align*}
						u(x , t) 
						&= D_2 F(x , ct) + \frac{1}{c} G(x , ct) \\
						&= \left. \frac{\partial}{\partial r} F(x , r) \right|_{r = ct} + \frac{1}{c} G(x , ct)
					\end{align*}
					
					\vspace*{4em}
					
					\item[\textbf{(2)}]. \underline{\textbf{求出F, G}}:$\left. \Big( r M_u(x ; r , t) \Big) \right|_{t = 0} 
					= F(x , r) , \,\, 
					\left. \Big( rM_{u}(x ; r , t) \Big)_t \right|_{t = 0} 
					= G(x , r)$. \\
					因此
					\begin{align*}
						F(x , r) 
						= r M_u(x ; r , t) \Big|_{t = 0} 
						= r \fint_{\partial B_r(x)} u(y , 0) \, dS_y 
						= r \fint_{\partial B_r(x)} f(y) \, dS_y
					\end{align*}
					对$F(x , r)$ 求$r$ 的偏导, 
					\begin{align*}
						\left. \frac{\partial}{\partial r} F(x , r) \right|_{r = ct} 
						&= \left. \frac{\partial}{\partial r} \left( r \cdot \fint_{\partial B_{r}(x)} f(y) \, dS_y \right) \right|_{r = ct} \\
						&= \frac{\partial}{\partial t} \left( t \cdot \fint_{\partial B_{ct}(x)} f(y) \, dS_y \right)
					\end{align*}
					同理可求得$G$
					\begin{align*}
						G(x , ct) 
						&= \left. \Big( rM_{u}(x ; r , t) \Big)_t \right|_{\substack{t = 0 \\ r = ct}} \\
						&=  \left. \left( r \fint_{\partial B_r(x)} u(y , t) \, dS_y \right) \right|_{\substack{t = 0 \\ r = ct}} \\
						&= \left. \left( r \fint_{\partial B_r(x)} g(y) \, dS_y \right) \right|_{r = ct} 
						= ct \cdot \fint_{\partial B_{ct}(x)} g(y) \, dS_y
					\end{align*}
				\end{enumerate}
				
				\newpage
				
				综上, 我们证明了若解存在, 则必为\textbf{式(\ref{4.2})}的形式:
				\begin{align*}
					u(x , t) 
					= D_2 F(x , ct) + G(x , ct) 
					&= \frac{\partial}{\partial t} \left( t \cdot \fint_{\partial B_{ct}(x)} f(y) \, dS_y \right) 
					+ t \cdot \fint_{\partial B_{ct}(x)} g(y) \, dS_y \\
					&= \frac{\partial}{\partial t} \left( \frac{1}{4\pi c^2 t} \int_{\partial B_{ct}(x)} f(y) \, dS_y \right) 
					+ \frac{1}{4\pi c^2 t} \int_{\partial B_{ct}(x)} g(y) \, dS_y 
				\end{align*}
				
				\vspace*{4em}
				
				\item[\underline{\textbf{Step 2}}]:\underline{\textbf{证明式(\ref{4.2})是解}}:下面分别证明$u$ 满足边界条件及波动方程:
				
				\vspace*{1em}
				
				\begin{enumerate}
					\item[\textbf{(1)}]. \underline{\textbf{证明边界条件}}:$u\Big|_{t = 0} = f , \,\, u_t \Big|_{t = 0} = g$:
					
					\vspace*{1em}
					
					对于$u\Big|_{t = 0}$, 
					\begin{align*}
						\frac{\partial}{\partial t} \left( \frac{1}{4\pi c^2 t} \int_{\partial B_{ct}(x)} f(y) \, dS_y \right) 
						&= \frac{\partial}{\partial t} \left( \frac{1}{4\pi c^2 t} \int_{\partial B_1} f(x + cty) \cdot (ct)^2 \, dS_y \right) \\
						&= \frac{\partial}{\partial t} \left( \frac{t}{4\pi} \int_{\partial B_1} f(x + cty) \, dS_y \right) \\
						&= \frac{1}{4\pi} \int_{\partial B_1} f(x + cty) \, dS_y 
						+ \frac{t}{4\pi} \int_{\partial B_1} \left( \sum_{i = 1}^3 D_{i} f(x + cty) \cdot c y_i \right) \, dS_y
					\end{align*}
					Letting $t \to 0^+$, we get 
					\begin{align*}
						\frac{\partial}{\partial t} \left( \frac{1}{4\pi c^2 t} \int_{\partial B_{ct}(x)} f(y) \, dS_y \right) \to f(x) \,\, \text{as} \,\, t \to 0^+
					\end{align*}
					Since
					\begin{align*}
						\lim_{t \to 0^+} \frac{1}{4\pi c^2 t} \int_{\partial B_{ct}(x)} g(y) \, dS_y  
						= 0
					\end{align*}
					Then $u \Big|_{t = 0} = f(x)$. 
					
					\newpage
					
					对于$u_t \Big|_{t = 0}$, 
					\begin{align*}
						\frac{\partial^2}{\partial t^2} \left( \frac{1}{4\pi c^2 t} \int_{\partial B_{ct}(x)} f(y) \, dS_y \right) 
						= R(x , t) + 2 \cdot \frac{1}{4 \pi} \int_{\partial B_{1}} \left( \sum_{i = 1}^3 D_{i} f(x + cty) \cdot c y_i \right) \, dS_y
					\end{align*}
					其中$R(x , t)$ 为求导后的其余项, 满足$R(x , t) \to 0$ as $t \to 0^+$. \\
					根据\textbf{控制收敛定理 (Dominated Convergence Theorem)\footnote{详见\textbf{《Real Analysis, Modern Techniques and Their Applications (Second Edition)》 -- Gerald B. Folland -- \\
								Theorem 2.24}}}, 
					\begin{align*}
						\lim_{t \to 0^+} \int_{\partial B_1} \Big( f_i(x + cty) \cdot c y_i \Big) \, dS_y 
						&= \int_{\partial B_1} \left( \lim_{t \to 0^+} f_i(x + cty) \cdot c y_i \right) \, dS_y \\
						&= f_i(x) \cdot \int_{\partial B_1} c y_i \, dS_y 
						= 0 , \,\, \forall i = 1 \sim n
					\end{align*}
					于是
					\begin{align*}
						\lim_{t \to 0^+} \frac{\partial^2}{\partial t^2} \left( \frac{1}{4\pi c^2 t} \int_{\partial B_{ct}(x)} f(y) \, dS_y \right) 
						= 0
					\end{align*}
					而由于
					\begin{align*}
						\frac{\partial}{\partial t} \left( \frac{1}{4\pi c^2 t} \int_{\partial B_{ct}(x)} g(y) \, dS_y \right) 
						\to g(x) \,\, \text{as} \,\, t \to 0^+
					\end{align*}
					因此$u_t \Big|_{t = 0} = g(x)$. 
					
					\vspace*{4em}
					
					\item[\textbf{(2)}]. \underline{\textbf{证明$u$ 满足波动方程}}:
					
					\vspace*{1em}
					
					只需证明$\dfrac{\partial}{\partial t} \left( \dfrac{1}{4\pi c^2 t} \int_{\partial B_{ct}(x)} f(y) \, dS_y \right)$ 和$\dfrac{1}{4\pi c^2 t} \int_{\partial B_{ct}(x)} g(y) \, dS_y$ 分别满足波动方程:
					
					\vspace*{1em}
					
					下面我们只证明前者, 后者可类似证明:即证
					\begin{align*}
						\frac{\partial^2}{\partial t^2} \left[ \frac{\partial}{\partial t}\left( \frac{t}{4\pi} \int_{\partial B_1} f(x + cty) \, dS_y \right) \right] 
						&= c^2 \sum_{i = 1}^3 \frac{\partial^2}{\partial x_{i}^2} \left[ \dfrac{t}{4\pi} \int_{\partial B_1} g(x + cty) \, dS_y \right] \\
						\text{Denoted as} \,\, \MakeUppercase{\rmnum{1}} &= \MakeUppercase{\rmnum{2}}
					\end{align*}
					将$\Rmnum{1}$ 中$\dfrac{\partial^2}{\partial t^2}$ 与$\dfrac{\partial}{\partial t}$ 交换, 
					\begin{align*}
						\Rmnum{1} 
						= \frac{\partial}{\partial t} \left( \frac{t}{4\pi} \int_{\partial B_1} \frac{\partial^2}{\partial t^2} f(x + cty) \, dS_y \right) 
						+ \frac{\partial}{\partial t} \left( 2 \cdot \frac{1}{4\pi} \int_{\partial B_1} \frac{\partial}{\partial t} f(x + cty) \, dS_y  \right)
					\end{align*}
					
					\newpage
					
					而
					\begin{align*}
						\frac{\partial}{\partial t} f(x + cty) 
						= \sum_{i = 1}^3 f_i \cdot cy_i 
						\hspace{1em} , \hspace{1em} 
						\frac{\partial}{\partial t} \left( \frac{\partial}{\partial t} f(x + cty) \right) 
						= \sum_{i = 1}^3 \sum_{j = 1}^3 f_{ij} \cdot cy_i \cdot cy_j
					\end{align*}
					因此要证$\Rmnum{1} = \Rmnum{2}$, 只需证:
					\begin{align*}
						\int_{\partial B_1} 
						\left[ 
						\sum_{i = 1}^3 \sum_{j = 1}^3 \Big( f_{ij} \cdot c^2 \cdot y_i \cdot y_j \Big) 
						+ \frac{2}{t} \sum_{i = 1}^3 \Big( f_i \cdot cy_i \Big) 
						\right] \, dS_y 
						= c^2 \int_{\partial B_1} \sum_{i = 1}^3 f_{ii}(x + cty) \, dS_y
					\end{align*}
					i.e. 
					\begin{align*}
						\int_{\partial B_1} \Big( \frac{\partial^2}{\partial t^2} f + \frac{2}{t} \cdot \frac{\partial}{\partial t} f \Big) \, dS_y 
						= \int_{\partial B_1} \Delta_x f \, dS_y
					\end{align*}
					根据\textbf{球面坐标轴Laplace算子与直角坐标轴Laplace算子的关系}, 记$\Delta_s$ 为球面上Laplace算子 
					\begin{align*}
						\Delta 
						&= \frac{\partial^2}{\partial r^2} + \frac{2}{r} \cdot \frac{\partial}{\partial r} + \frac{1}{r^2} \Delta_s \\
						\Delta_s 
						&= \frac{1}{\sqrt{g}} \cdot \frac{\partial}{\partial z_i} \left( \sqrt{g} \cdot g^{ij} \cdot \frac{\partial}{\partial z_j} \right)
					\end{align*}
					而$\int_{\partial B_1} \dfrac{1}{r^2} \Delta_s f = 0$, 从而
					\begin{align*}
						\int_{\partial B_1} \Big( \frac{\partial^2}{\partial t^2} f + \frac{2}{t} \cdot \frac{\partial}{\partial t} f \Big) \, dS_y 
						= \int_{\partial B_1} \Delta_x f \, dS_y
					\end{align*}
					故得证. 
				\end{enumerate}
			\end{enumerate}
		\end{proof}
	\end{thm}















	%  ############################
	\ifx\allfiles\undefined
\end{document}
\fi