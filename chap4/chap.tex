\ifx\allfiles\undefined
\documentclass[12pt, a4paper,oneside, UTF8]{ctexbook}
\usepackage[dvipsnames]{xcolor}
\usepackage{mathtools}   % 数学公式(mathtools 是 amsmath 的上位替代)
\usepackage{amsthm}    % 定理环境
\usepackage{amssymb}   % 更多公式符号
\usepackage{graphicx}  % 插图
%\usepackage{mathrsfs}  % 数学字体
%\usepackage{newtxtext,newtxmath}
%\usepackage{arev}
\usepackage{kmath,kerkis}
\usepackage{newtxtext}
\usepackage{bbm}
\usepackage{enumitem}  % 列表
\usepackage{geometry}  % 页面调整
%\usepackage{unicode-math}
\usepackage[colorlinks,linkcolor=black]{hyperref}

\usepackage{wrapfig}


\usepackage{ulem}	   % 用于更多的下划线格式,
					   % \uline{}下划线,\uuline{}双下划线,\uwave{}下划波浪线,\sout{}中间删除线,\xout{}斜删除线
					   % \dashuline{}下划虚线,\dotuline{}文字底部加点


\graphicspath{ {flg/},{../flg/}, {config/}, {../config/} }  % 配置图形文件检索目录
\linespread{1.5} % 行高

% 页码设置
\geometry{top=25.4mm,bottom=25.4mm,left=20mm,right=20mm,headheight=2.17cm,headsep=4mm,footskip=12mm}

% 设置列表环境的上下间距
\setenumerate[1]{itemsep=5pt,partopsep=0pt,parsep=\parskip,topsep=5pt}
\setitemize[1]{itemsep=5pt,partopsep=0pt,parsep=\parskip,topsep=5pt}
\setdescription{itemsep=5pt,partopsep=0pt,parsep=\parskip,topsep=5pt}

% 定理环境
% ########## 定理环境 start ####################################
\theoremstyle{definition}
\newtheorem{defn}{\indent 定义}[section]

\newtheorem{lemma}{\indent 引理}[section]    % 引理 定理 推论 准则 共用一个编号计数
\newtheorem{thm}[lemma]{\indent 定理}
\newtheorem{corollary}[lemma]{\indent 推论}
\newtheorem{criterion}[lemma]{\indent 准则}

\newtheorem{proposition}{\indent 命题}[section]
\newtheorem{example}{\indent \color{SeaGreen}{例}}[section] % 绿色文字的 例 ,不需要就去除\color{SeaGreen}{}
\newtheorem*{rmk}{\indent \color{red}{注}}

% 两种方式定义中文的 证明 和 解 的环境:
% 缺点:\qedhere 命令将会失效【技术有限,暂时无法解决】
\renewenvironment{proof}{\par\textbf{证明.}\;}{\qed\par}
\newenvironment{solution}{\par{\textbf{解.}}\;}{\qed\par}

% 缺点:\bf 是过时命令,可以用 textb f等替代,但编译会有关于字体的警告,不过不影响使用【技术有限,暂时无法解决】
%\renewcommand{\proofname}{\indent\bf 证明}
%\newenvironment{solution}{\begin{proof}[\indent\bf 解]}{\end{proof}}
% ######### 定理环境 end  #####################################

% ↓↓↓↓↓↓↓↓↓↓↓↓↓↓↓↓↓ 以下是自定义的命令  ↓↓↓↓↓↓↓↓↓↓↓↓↓↓↓↓

% 用于调整表格的高度  使用 \hline\xrowht{25pt}
\newcommand{\xrowht}[2][0]{\addstackgap[.5\dimexpr#2\relax]{\vphantom{#1}}}

% 表格环境内长内容换行
\newcommand{\tabincell}[2]{\begin{tabular}{@{}#1@{}}#2\end{tabular}}

% 使用\linespread{1.5} 之后 cases 环境的行高也会改变,重新定义一个 ca 环境可以自动控制 cases 环境行高
\newenvironment{ca}[1][1]{\linespread{#1} \selectfont \begin{cases}}{\end{cases}}
% 和上面一样
\newenvironment{vx}[1][1]{\linespread{#1} \selectfont \begin{vmatrix}}{\end{vmatrix}}

\def\d{\textup{d}} % 直立体 d 用于微分符号 dx
\def\R{\mathbb{R}} % 实数域
\def\N{\mathbb{N}} % 自然数域
\def\C{\mathbb{C}} % 复数域
\def\Z{\mathbb{Z}} % 整数环
\def\Q{\mathbb{Q}} % 有理数域
\newcommand{\bs}[1]{\boldsymbol{#1}}    % 加粗,常用于向量
\newcommand{\ora}[1]{\overrightarrow{#1}} % 向量

% 数学 平行 符号
\newcommand{\pll}{\kern 0.56em/\kern -0.8em /\kern 0.56em}

% 用于空行\myspace{1} 表示空一行 填 2 表示空两行  
\newcommand{\myspace}[1]{\par\vspace{#1\baselineskip}}

%s.t. 用\st就能打出s.t.
\DeclareMathOperator{\st}{s.t.}

%罗马数字 \rmnum{}是小写罗马数字, \Rmnum{}是大写罗马数字
\makeatletter
\newcommand{\rmnum}[1]{\romannumeral #1}
\newcommand{\Rmnum}[1]{\expandafter@slowromancap\romannumeral #1@}
\makeatother
\begin{document}
	% \title{{\Huge{\textbf{$Partial \,\, Differential \,\, Equations$}}}\footnote{参考书籍:\\
			\hspace*{4em} \textbf{《Partial Differential Equations》 -- Lawrence C. Evans} \\
			\hspace*{4em} \textbf{《Partial Differential Equations》 -- Fritz John} \\
			\hspace*{4em} \textbf{《数学物理方程讲义 (第二版)》--  姜礼尚、陈亚浙、刘西垣、易法槐} 
			}}
\author{$-TW-$}
\date{\today}
\maketitle                   % 在单独的标题页上生成一个标题

\thispagestyle{empty}        % 前言页面不使用页码
\begin{center}
	\Huge\textbf{序}
\end{center}


\vspace*{3em}
\begin{center}
	\large{\textbf{天道几何,万品流形先自守;}}\\
	
	\large{\textbf{变分无限,孤心测度有同伦。}}
\end{center}

\vspace*{3em}
\begin{flushright}
	\begin{tabular}{c}
		\today \\ \small{\textbf{长夜伴浪破晓梦,梦晓破浪伴夜长}}
	\end{tabular}
\end{flushright}


\newpage                      % 新的一页
\pagestyle{plain}             % 设置页眉和页脚的排版方式(plain:页眉是空的,页脚只包含一个居中的页码)
\setcounter{page}{1}          % 重新定义页码从第一页开始
\pagenumbering{Roman}         % 使用大写的罗马数字作为页码
\tableofcontents              % 生成目录

\newpage                      % 以下是正文
\pagestyle{plain}
\setcounter{page}{1}          % 使用阿拉伯数字作为页码
\pagenumbering{arabic}
\setcounter{chapter}{0}    % 设置 -1 可作为第零章绪论从第零章开始 
	\else
	\fi
	%  ############################ 正文部分
\chapter{线性算子与线性泛函}
	
\section{线性算子$\&$ 线性泛函}
	这一节我们主要来给出\textbf{线性算子}及\textbf{线性泛函}相关概念的定义. \textbf{线性算子}为高等代数中学过的\textbf{线性变换 (线性映射)}的推广, 即更多的在\textbf{无穷维线性空间}上进行讨论. 而\textbf{线性泛函}是\textbf{线性算子}的一个特例, 即将线性泛函的陪域取作数域, 相当于定义域较大的函数, 即为函数的推广. 
	
	\vspace{1em}
	
	\begin{defn}\label{def 4.1.1}
		设$X , Y$ 为定义在数域$\mathbb{K}$ 上的线性空间. 若映射$T : X \longrightarrow Y$ 为线性映射, 即
		\[ T(\alpha x + \beta y) = \alpha T(x) + \beta T(y) , \,\, \forall x , y \in X , \,\, \forall \alpha , \beta \in \mathbb{K} \]
		则称$T$ 为$X$ 到$Y$ 上的一个\underline{\textcolor{blue}{\textbf{线性算子}}}. 特别地, 若$Y = \mathbb{K} \in \{ \R , \C \}$, 则称$T$ 为\underline{\textcolor{blue}{\textbf{线性泛函}}}. 
		
		\vspace{4em}
		
		\begin{rmk}
			\begin{itemize}
				\item 此处我们沿用高代与范畴论中的记号, 将从线性空间$X$ 到$Y$ 上的所有\textbf{线性算子}构成的空间记作\textbf{$Hom(X , Y)$}. 不难说明$Hom(X , Y)$ 也是个定义在数域$\mathbb{K}$ 上的\textbf{线性空间}. 
				
				\vspace*{4em}
				
				\item 回顾高代中有关\textbf{有限维线性空间的对偶空间}的结论:
				\begin{center}
					\textbf{数域$\mathbb{K}$ 上有限维线性空间的所有线性泛函构成空间 (对偶空间)同构于$\mathbb{K}^n$}. 
				\end{center} 
				
				\vspace*{2em}
				
				\begin{proof}
					设$X$ 为数域$\mathbb{K}$ 上$n$ 维线性空间, $\{ e_i \}_{i = 1}^n \subset X$ 为一组基. \\
					Then for $\forall f \in Hom(X , \mathbb{K})$, 
					\[ f(x) = f \left( \sum_{i = 1}^n x_i e_i \right) = \sum_{i = 1}^n x_i f(e_i) , \,\, \forall x = \sum_{i = 1}^n x_i e_i \in X \]
					Consider the mapping 
					\begin{align}
						T : Hom(X , \mathbb{K}) &\longrightarrow \mathbb{K}^n \\
						f &\longmapsto \Big( f(e_1) , \,\, \cdots , \,\, f(e_n) \Big)
					\end{align}
					It's not hard to prove that $T$ is an isomorphism between $Hom(X , \mathbb{K})$ and $\mathbb{K}^n$, i.e. 
					\[ Hom(X , \mathbb{K}) \cong \mathbb{K}^n \]
				\end{proof}
			\end{itemize}
		\end{rmk}
	\end{defn}
	
	\vspace*{6em}
	
	对于线性空间$X$ 到$Y$ 上的线性算子$Hom(X , Y)$, 为了考虑其连续性, 下面将$X , Y$ 限制为$B^*$ 空间, 给出\textbf{连续算子}的定义. 
	
	\vspace*{1em}
	
	\begin{defn}\label{def 4.1.2}
		设$X , Y \in B^{*}$, $T \in Hom(X , Y)$. 若$T$ 连续, 即在$X$ 中每点处连续, 则称$T$ 为\underline{\textcolor{blue}{\textbf{连续算子}}}, 记$X$ 到$Y$ 的连续算子全体为\underline{\textcolor{blue}{\textbf{$\, L(X , Y)$}}}. 特别地, 若$Y = \mathbb{K} \in \{ \R , \C \}$, 记\underline{\textcolor{blue}{\textbf{$\, X^* = L(X , \mathbb{K})$}}}. 
		
		\vspace*{4em}
		
		\begin{rmk}
			\begin{itemize}
				\item 事实上, 此处记号$X^*$ 是对高代中\textbf{对偶空间}概念的推广, 即\uwave{当$X$ 为\textbf{有限维}线性空间时, $X^* = L(X , \mathbb{K})$ 就是$X$ 上的对偶空间}, 故更多讨论无穷维的情况. 这一点由下述命题保证:
				\begin{center}
					\textbf{有限维$B^*$ 空间$(X , \Vert \cdot \Vert)$ 上的线性函数$f : X \longrightarrow \mathbb{K}$ 连续}. 
				\end{center}
				
				\vspace*{1em}
				
				\begin{proof}
					根据\textbf{有限维$B^*$ 空间范数的等价性 (Thm \ref{thm 2.2.2})} 容易证明, 于是$Hom(X , \mathbb{K}) = X^*$.  
				\end{proof}
				
				\vspace*{6em}
				
				\item 对于$\forall T \in Hom(X , \mathbb{K})$, 根据$T$ 的线性性, 不难得到$T \in L(X , Y) \,\, \Leftrightarrow \,\, T$ 在$x = 0$ 处连续. 
			\end{itemize}
		\end{rmk}
	\end{defn}
	
	\newpage
	
	下面再给出\textbf{有界算子}的概念. 
	
	\vspace*{1em}
	
	\begin{defn}\label{def 4.1.3}
		设$X , Y \in B^*$ 且$T \in Hom(X , Y)$. 如果$T$ 将任何有界集映为有界集, 则称$T$ 为\underline{\textcolor{blue}{\textbf{有界 (线性)算子}}}. 
		
		\vspace*{4em}
		
		\begin{rmk}
			此处给出$T$ 有界的几个\textbf{等价定义}, 即对于$\forall T \in Hom(X , Y)$, 
			\begin{center}
				\textbf{$T$ 有界 $\,\, \Leftrightarrow \,\, $ 单位球 (面)的像有界 $\,\, \Leftrightarrow \,\, \exists M > 0 , \,\, \st \Vert T x \Vert \leq M \, \Vert x \Vert , \,\, \forall x \in X$}
			\end{center}
			
			\vspace*{2em}
			
			\begin{proof}
				\begin{itemize}
					\item $T$ 有界 $\,\, \Leftrightarrow \,\, $ 单位球 (面)的像有界:必要性显然. 下面证充分性$\Leftarrow$:$\exists M > 0$, $\st$
					\[ \Vert Tx \Vert \leq M \, \Vert x \Vert = M , \,\, \forall x \in B(0 , 1) \]
					$\forall A \subset X$ bounded, i.e. $\exists r > 0$, $\st A \subset B(0 , r)$. Then by the linearity of $T \in Hom(X , Y)$, 
					\[ \Vert T x \Vert 
					= \Big\Vert T \Big( \Vert x \Vert \cdot \frac{x}{\Vert x \Vert} \Big) \Big\Vert 
					= \Vert x \Vert \cdot \Big\Vert T \Big( \frac{x}{\Vert x \Vert} \Big) \Big\Vert 
					\leq r \cdot M < \infty , \,\, \forall x \in A \]
					Therefore, $T(A) \subset \mathbb{K}$ bounded. $T$ bounded. 
					
					\vspace*{8em}
					
					\item 单位球 (面)的像有界 $\,\, \Leftrightarrow \,\, \exists M > 0 , \,\, \st \Vert T x \Vert \leq M \, \Vert x \Vert , \,\, \forall x \in X$:\\
					充分性显然. 下面证明必要性$\Rightarrow$:$\exists M > 0$, $\st$
					\[ \Vert Tx \Vert \leq M , \,\, \forall x \in \partial B(0 , 1) \]
					Then for $\forall x \in X$, 
					\[ \Big\Vert T\Big( \frac{x}{\Vert x \Vert} \Big) \Big\Vert 
					= \frac{\Vert Tx \Vert}{\Vert x \Vert}
					\leq M , \,\, \forall x \in M \]
					i.e. $\Vert Tx \Vert \leq M \, \Vert x \Vert , \,\, \forall x \in X$.
				\end{itemize}
			\end{proof}
		\end{rmk}
	\end{defn}


	%  ############################
	\ifx\allfiles\undefined
\end{document}
\fi