\ifx\allfiles\undefined
\input{../config/config}
\begin{document}
	% \input{../config/cover} 
	\else
	\fi
	%  ############################ 正文部分
\appendix
\chapter{$Supplementary \,\, Content$}\label{appendix A}

\section{区域边界的光滑性}
	下面我们来给出\textbf{区域边界的光滑性}的定义.
	
	\begin{defn}\label{def A.1.1}
		Suppose $\Omega \subset \R^n$ is open, $\partial \Omega \neq \varnothing$. 如果对于$\forall p \in \partial \Omega$, $\exists p$ 的邻域$U$, $\st$ 在适当的空间直角坐标系下, 
		\begin{align}
			\partial \Omega \cap U &= \{ (x^{'} , x^{n}) \in \R^{n - 1} \times \R \mid x^{'} \in D , \,\, x^n = \varphi(x^{'}) \} \\
			\Omega \cap U &= \{ (x^{'} , x^n) \in \R^{n - 1} \times \R \mid x^{'} \in D , \,\, x^n > \varphi(x^{'}) \} \cap U
		\end{align}
		where $\varphi \in C^{k}(\Omega) , \,\, D \subset \R^{n - 1}$ open. 我们称\underline{\textcolor{blue}{\textbf{$\partial \Omega \in C^k$}}}, 这里$k = 0 , 1 , 2 , \cdots$ 或$\infty$.
		
		\begin{figure}[thbp!]
			\centering
			\includegraphics[width=0.5\linewidth]{figure/A.1-1}
			\caption{边界的光滑性}
			\label{pic : A.1-1} % 添加图像引用标签
		\end{figure}
		
		\begin{rmk}
			在\textbf{定义 \ref{def A.1.1}} 中, 设$k \leq 1$, $\,\, x_{0}^{'} \in \partial \Omega$, $\,\, p = (x_{0}^{'} , \varphi(x_{0}^{'}))$. 记
			\[ n_p = \frac{(\nabla \varphi(x_{0}^{'}) , -1)}{\sqrt{1 + \left| \nabla \varphi(x_{0}^{'}) \right|^2}} \]
			称$n_p$ 为$\partial \Omega$ 在$p$ 点的\underline{\textcolor{blue}{\textbf{单位外法向}}}. $n_p$ 的定义与空间直角坐标系的选择无关. 记
			\begin{align}
				n : \partial \Omega &\longrightarrow \R^n \\
				p &\longmapsto n(p) = n_p 
			\end{align}
			称$n$ 为$\partial \Omega$ 的\underline{\textcolor{blue}{\textbf{单位外法向}}}. 因为$\partial \Omega \in C^k$, $k \geq 1$, 所以$n \in C(\partial \Omega ; \R^n)$.
		\end{rmk}
	\end{defn}

\newpage

\section{含参变量的积分}
	在分析中, 我们常遇到如下形式的积分定义的函数:
	\[ \varphi(x) = \int_{c}^d f(x , t) \, dt , \,\, x \in [a , b] \]
	上式右端的积分称为\textbf{含参变量 (参变量为$x$) 的积分}. 下面我们来给出由含参变量的积分所定义的函数的\textbf{连续性}、\textbf{可积性}和\textbf{可微性}. 
	
	\begin{thm}\label{thm A.2.1}
		设$f \in C([a , b] \times [c , d])$. 记$\varphi : [a , b] \longrightarrow \R$, 
		\[ \varphi(x) = \int_{c}^d f(x , t) \, dt , \,\, \forall x \in [a , b] \]
		则
		\begin{enumerate}
			\item[(\rmnum{1})] $\varphi \in C[a , b]$. 
			
			\item[(\rmnum{2})] If $D_{1}f \in C([a , b] \times [c , d])$, then $\varphi \in C^{1}([a , b])$ and 
			\[ \varphi^{'}(x) = \int_{c}^d D_{1}f(x , t) \,  dt , \,\, \forall x \in [a , b] \]
			
			\item[(\rmnum{3})] $\varphi$ Riemann可积, 并且
			\[ \int_{a}^b \varphi = \int_{c}^d \left( \int_{a}^b f(x , t) \, dx \right) \, dt \]
		\end{enumerate}
	
		\vspace{4em}
		
		\begin{rmk}
			\begin{itemize}
				\item (\rmnum{1})中$\varphi \in C[a , b]$ 表明了\textbf{积分与极限可交换次序}, 即
				\[ \lim_{x \to x_0} \int_{c}^d f(x , t) \, dt = \int_{c}^d \lim_{x \to x_0} f(x , t) \, dt = \int_{c}^d f(x_0 , t) \, dt , \,\, \forall x_0 \in [a , b] \]
				同理, (\rmnum{2})和(\rmnum{3})也分别告诉我们\textbf{求导和积分、积分和积分之间可交换次序}.
				
				\vspace{4em}
				
				\item 根据数学归纳法, 如果$f \in C^{k}([a , b] \times [c , d])$, 则$\varphi \in C^{k}[a , b]$, 
				\[ \varphi^{(i)}(x) = \int_{c}^d \frac{d^i}{d x^i} f(x , t) \, dt , \,\, \forall i = 1 \sim k \]
				即\textbf{任意$k$ 阶内积分与求导可交换次序}, 即
				\[ \frac{d^i}{dx^{i}} \int_{c}^d f(x , t) \, dt = \int_{c}^d \frac{d^i}{d x^i} f(x , t) \, dt , \,\, \forall i = 1 \sim k \]
			\end{itemize}
		\end{rmk}
	\end{thm}

\newpage

\section{含参变量的广义积分的一致收敛}
	为了考察\textbf{含参变量的广义积分的连续性、可积性和可微性}, 我们先来给出\textbf{含参变量广义积分的一致收敛}的概念.
	
	\begin{defn}\label{def A.3.1}
		设$f : D \times [a , \infty) \longrightarrow \R$, 对于$\forall x \in D$, $b > a$, $f(x , \cdot)$ 在$[a , b]$ 上Riemann可积. 如果对于$\forall x \in D$, 积分$\int_{a}^{\infty} f(x , t) \, dt$ 收敛, 并且对于$\forall \epsilon > 0$, $\exists M > a$, $\st$
		\[ \left| \int_{A}^{\infty} f(x , t) \, dt \right| \leq \epsilon , \,\, \forall A > M , \,\, \forall x \in D \]
		称\underline{\textcolor{blue}{\textbf{积分$\int_{a}^\infty f(x , t) \, dt$ 关于$x$  一致收敛}}}. 
		
		\vspace{4em}
		
		\begin{rmk}
			\begin{itemize}
				\item 设$A > a$. Let
				\begin{align}
					\varphi(x) &= \int_{a}^{\infty} f(x , t) \, dt , \,\, x \in D \\
					\widetilde{\varphi}(x) &= \int_{a}^A f(x , t) \, dt , \,\, x \in D
				\end{align}
				由定义, 如果积分$\int_{a}^\infty f(x , t) \, dt$ 关于$x$ 一致收敛, 则当$A$ 充分大时, 积分$\int_{a}^{\infty} f(x , t) \, dt$ 关于$x$ 一致地小. 此时
				\[ \varphi(x) = \int_{a}^A f(x , t) \, dt + \int_{A}^{\infty} f(x , t) \, dt \approx \widetilde{\varphi}(x) \]
				所以我们可以期望
				\begin{center}
					\textbf{一致收敛的}由含参变量的无穷限积分所定义的函数$\varphi$ \\
					与 \\
					由含参变量的常义积分所定义的$\widetilde{\varphi}$ 有相同的性质.
				\end{center}
				
				\vspace{4em}
				
				\item $\forall a_1 > a$, 
				\begin{center}
					$\int_{a}^{\infty} f(x , t) \, dt$ 一致收敛 $\,\, \Leftrightarrow \,\, \int_{a_1}^{\infty} f(x , t) \, dt$ 一致收敛
				\end{center}
			\end{itemize}
		\end{rmk}
	\end{defn}

\newpage

	下面给出判定积分$\int_{a}^{\infty} f(x , t) \, dt$ 是否一致收敛的一些判定准则. 

	\begin{proposition}\label{prop A.3.1}
		\textbf{[Cauchy收敛原理]}. \\
		设$f : D \times [a , \infty) \longrightarrow \R$, 对于$\forall x \in D$, $b > a$, $f(x , \cdot)$ 在$[a , b]$ 上Riemann可积, 则
		\begin{center}
			积分$\int_{a}^\infty f(x , t) \, dt$ 一致收敛 $\,\, \Leftrightarrow \,\,$ 对于$\forall \epsilon > 0$, $\exists M > a$, $\st$
			\[ \left| \int_{A_1}^{A_2} f(x , t) \, dt \right| \leq \epsilon , \,\, \forall A_2 > A_1 \geq M , \,\, \forall x\ in D \]
		\end{center}
	\end{proposition}

	\vspace{8em}

	\begin{proposition}\label{prop A.3.2}
		\textbf{[比较判别法]}. \\
		设$f : D \times [a , \infty) \longrightarrow \R$, 对于$\forall x \in D$, $b > a$, $f(x , \cdot)$ 在$[a , b]$ 上Riemann可积. 设$g : [a , \infty) \longrightarrow \R$, $g \geq 0$, 对于$\forall b > a$, $g$ 在$[a , b]$ 上Riemann可积. 如果
		\[ \left| f(x  ,t) \right| \leq g(t) , \,\, \forall (x ,t) \in D \times [a , \infty) \]
		并且$\int_{a}^{\infty} g(t) \, dt < \infty$, 则积分$\int_{a}^{\infty} f(x , t) \, dt$ 一致收敛.
	\end{proposition}

	\vspace{8em}
	
	\begin{proposition}\label{prop A.3.3}
		\textbf{[Dirichlet判别法]}. \\
		设$f , g : D \times [a , \infty) \longrightarrow \R$, 对于$\forall x \in D$, $b > a$, $f(x , \cdot) , g(x , \cdot)$ 在$[a , b]$ 上Riemann可积. 设对于$\forall x \in D$, $g(x , \cdot)$ 单调. 如果
		\begin{enumerate}
			\item[(\rmnum{1})] $\forall \epsilon > 0$, $\exists M > a$, $\st$
			\[ \left| g(x , t) \right| \leq \epsilon , \,\, \forall t \geq M , x \in D \]
			
			\item[(\rmnum{2})] $\exists L \in \R$, $\st$
			\[ \left| \int_{a}^b f(x , t) \, dt \right| \leq L , \,\, \forall b \geq a , \,\, \forall x \in D \]
		\end{enumerate}
		则积分$\int_{a}^{\infty} f(x , t) g(x , t) \, dt$ 一致收敛.
	\end{proposition}
	
	\newpage
	
	\begin{proposition}\label{prop A.3.4}
		\textbf{[Abel判别法]}. \\
		设$f , g : D \times [a , \infty) \longrightarrow \R$, 对于$\forall x \in D$, $b > a$, $f(x , \cdot) , g(x , \cdot)$ 在$[a , b]$ 上Riemann可积. 设对于$\forall x \in D$, $g(x , \cdot)$ 单调. 如果
		\begin{enumerate}
			\item[(\rmnum{1})] $\exists L \in \R$, $\st$
			\[ \left| g(x , t) \right| \leq L , \,\, \forall (x  , t) \in D \times [a , \infty) \]
			
			\item[(\rmnum{2})] 积分$\int_{a}^\infty f(x , t) \, dt$ 一致收敛
		\end{enumerate}
		则积分$\int_{a}^{\infty} f(x , t) g(x , t) \, dt$ 一致收敛.
	\end{proposition}

\newpage

\section{含参变量的广义积分的性质}
	下面讨论由含参变量的广义积分所定义的函数的\textbf{连续性}、\textbf{可积性}和\textbf{可微性}, 即\textbf{积分与极限交换次序}、\textbf{积分与积分交换次序}、\textbf{积分与求导交换次序}的问题. 
	
	\vspace{1em}
	
	首先给出\textbf{一致收敛}时积分$\int_{c}^{\infty} f(x , t) \, dt$ 的\textbf{连续性}和\textbf{可积性}. 
	
	\begin{proposition}\label{prop A.4.1}
		设$f \in C([a , b] \times [c , \infty])$. 设积分$\int_{c}^{\infty} f(x , t) \, dt$ 关于$x$ 一致收敛. 记
		\[ \varphi(x) = \int_{c}^{\infty} f(x , t) \, dt , \,\, \forall x \in [a , b] \]
		则
		\begin{enumerate}
			\item[(\rmnum{1})] $\varphi \in C[a , b]$. 
			
			\item[(\rmnum{2})] 积分$\int_{c}^{\infty} \left( \int_{a}^b f(x , t) \, dx \right) \, dt$ 收敛, 并且
			\[ \int_{a}^b \varphi = \int_{c}^{\infty} \left( \int_{a}^b f(x , t) \, dx \right) \, dt \]
		\end{enumerate}
	\end{proposition}

	\vspace{12em}
	
	下面再给出积分$\int_{c}^{\infty} f(x , t) \, dt$ 的\textbf{可微性}.
	
	\begin{proposition}\label{prop A.4.2}
		设$f : [a , b] \times [c , d] \longrightarrow \R , \,\, f , D_{1}f \in C([a , b] \times [c  ,d])$. 如果
		\begin{center}
			\begin{enumerate}
				\item[(\rmnum{1})] $\exists x_0 \in [a , b]$, $\st$ 积分$\int_{c}^{\infty} f(x_0 , t) \, dt$ 收敛.
				
				\item[(\rmnum{2})] 积分$\int_{c}^{\infty} D_{1}f(x , t) \, dt$ 关于$x$ 一致收敛.
			\end{enumerate}
		\end{center}
		则积分$\int_{c}^{\infty} f(x , t) \, dt$ 关于$x$ 一致收敛. 记
		\[ \varphi(x) = \int_{c}^{\infty} f(x , t) \, dt , \,\, \forall x \in [a , b] \]
		则$\varphi \in C^{1}[a , b]$, 并且
		\[ \varphi^{'}(x) = \int_{c}^{\infty} D_{1}f(x , t) \, dt , \,\, \forall x \in [a , b] \]
	\end{proposition}

	\newpage
	
	根据\textbf{命题 \ref{prop A.4.2}}, 运用数学归纳法可得到如下推论.
	
	\begin{corollary}\label{cor A.4.1}
		设$f \in C^{k}([a , b] \times [c , d])$, $k \in \N$. 设积分
		\[ \int_{c}^{\infty} \frac{\partial^i}{\partial x^i} f(x , t) \, dt , \,\, i = 0 \sim k \]
		均关于$x$ 一致收敛. 记
		\[ \varphi(x) = \int_{c}^{\infty} f(x , t) \, dt , \,\, \forall x \in [a , b] \]
		则$\varphi \in C^{k}[a , b]$, 且
		\[ \varphi^{(i)}(x) = \int_{c}^{\infty} \frac{\partial^i}{\partial x^i} f(x , t) \, dt , \,\, \forall x \in [a , b] \]
	\end{corollary}

\newpage

\section{卷积的性质}
	首先给出卷积的定义.
	\begin{defn}\label{def A.5.1}
		设$f , g \in C(\R^n)$, 并且$f$ 或$g$ 有紧支集. 定义
		\begin{align}
			f * g : \R^n &\longrightarrow \R \\
			x &\longmapsto \int_{\R^n} f(x - y) g(y) \, dy
		\end{align}
		称$f*g$ 为\underline{\textcolor{blue}{\textbf{$f , g$ 的卷积}}}. 
		
		\begin{rmk}
			不难证明卷积具有\textbf{对称性}, 即
			\[ f*g = g*f \]
		\end{rmk}
	\end{defn}

	\vspace{4em}
	
	下面给出卷积的性质. 
	\begin{proposition}\label{prop A.5.1}
		If $f \in C^{k}(\R^n)$, $g \in C_{c}^{l}(\R^n)$, then $f * g \in C^{k + l}(\R^n)$ and
		\[ \frac{d}{dx}(f * g) = \left( \frac{d}{dx} f \right) * g = f * \left( \frac{d}{dx} g \right) \]
		where $k = 0 , 1 , 2 , \cdots$ or $\infty$, and $l = 1 , 2 , \cdots$ or $\infty$.
		
		\vspace{6em}
		
		\begin{proof}
			下面证明$n = 1$ 的情形, 可推广至高维情形. \\
			Since $g \in C_{c}^{l}(\R)$, then $\exists M , m \in \R$, $\st supp \, g \subset [m , M]$. Thus
			\[ f*g (x) 
			= \int_{\R} f(t) g(x - t) \, dt 
			= \int_{x - M}^{x - m} f(t) g(x - t) \, dt , \,\, \forall x \in \R \]
			Fix $x \in \R$. Since
			\begin{align}
				\left| \frac{f*g (x + h) - f*g(x)}{h} - f* g^{'}(x) \right| 
				\leq \int_{x - M}^{x - m} \left| f(t) \right| \left| g^{'}(\xi) - g^{'}(x) \right| \, dt
			\end{align}
			where $\xi$ stands between $x - t$ and $x - t + h$. \\
			Since $g^{'} \in C_{c}(\R)$ converges uniformly on $\R$, then for $\left| h \right| > 0$ small enough, 
			\[ \left| g^{'}(\xi) - g^{'}(x) \right| \leq \epsilon , \,\, \forall t \in \R \]
			Thus
			\begin{align}
				\left| \frac{f*g (x + h) - f*g(x)}{h} - f* g^{'}(x) \right| 
				\leq \epsilon \int_{x - M}^{x - m} \left| f(t) \right| \, dt
			\end{align}
			Since $f$ is continous, $[x - M , x - m] \subset \R$ is compact, then $\exists A_x > 0$, $\st$
			\[ \left| \frac{f*g (x + h) - f*g(x)}{h} - f* g^{'}(x) \right| \leq A_x \epsilon \]
			Letting $\epsilon \to 0$, we get
			\[ \frac{d}{dx}(f * g)(x) = f * \left( \frac{d}{dx} g \right) (x) , \,\, \forall x \in \R  \]
			Similarly, we get
			\[ \frac{d}{dx}(f * g) = \left( \frac{d}{dx} f \right) * g = f * \left( \frac{d}{dx} g \right) \]
		\end{proof}
	\end{proposition}
	
	







	%  ############################
	\ifx\allfiles\undefined
\end{document}
\fi